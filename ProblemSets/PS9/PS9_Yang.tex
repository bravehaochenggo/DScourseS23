\documentclass{article}

% Language setting
% Replace `english' with e.g. `spanish' to change the document language
\usepackage[english]{babel}

% Set page size and margins
% Replace `letterpaper' with `a4paper' for UK/EU standard size
\usepackage[letterpaper,top=2cm,bottom=2cm,left=3cm,right=3cm,marginparwidth=1.75cm]{geometry}

% Useful packages
\usepackage{amsmath}
\usepackage{graphicx}
\usepackage[colorlinks=true, allcolors=blue]{hyperref}

\title{Problem Set 9}
\author{Haocheng Yang}

\begin{document}
\maketitle


\section{Dimension of Training Data and Variables of Original Dataset}

The training dataset contains 404 observations with 14 variables.

The original housing dataset contains 14 variables with 506 observations

\section{LASSO: optimal value of Lamda and RMSE }

The optimal value of Lamda is 0.0676;
The out of sample RMSE is 0.170;

\section{Ridge: optimal value of Lamda and RMSE}
The optimal value of Lamda is 0.0748;
The out of sample RMSE is 0.173;

\section{Question 10}

I don't think that it is a good way to estimate using simple linear regression model on a data set that has more columns than rows. It might lead to model overfitting.

Regarding the bias-variance tradeoff, a model with low bias will fit the training data well, but may overfit and perform poorly on new data.

In the previous questions, we used cross-validation to tune the models and estimate their performance on new data. The optimal RMSE values obtained suggest that the Lasso model with the tuned lambda parameter strikes a good balance between bias and variance. It is able to achieve low out-of-sample RMSE without overfitting the training data.

\end{document}